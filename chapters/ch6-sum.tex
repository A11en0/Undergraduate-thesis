
\chapter{总结与展望}

\section{工作总结}
当下,随着信息与通信技术的发展,各种网络平台不断壮大,信息随时产生,海量短文本数据汇入信
息的洋流。这些数据蕴含丰富的信息,有很高的研究价值,因此高效挖掘这些数据的内涵变得愈发重要。短
文本分类工作一直是数据挖掘领域研究的热点,它有着不用一般文本数据的特点,即语义缺失、高维稀
疏等。并且,短文本数据流还会随时间迁移产生概念漂移现象。这些特点使得传统算法的对其应用的效果较差。因此,本文借助了文本扩展的方法,缓解短文本
的语义信息缺乏的问题,并提出一种基于块的集成分类模型,考虑对概念漂移的检测,对短文本数据流
分类进行了以下的研究:

1. 本文第一章介绍了短文本数据流分类的研究背景及意义,给出短文本数据流分类面临的问题和挑战,介绍国内外研究人员对短文本分类问题研究的现状,对已存在的问题提出的解决方式。接
着阐述了本文主要研究的内容,即设计一个集成分类算法和可视化的数据挖掘平台。给出论文的组织框
架,分别简要介绍了每个章节的主要内容。

2. 第二章为相关技术和理论的介绍,给出了文本挖掘的一般步
骤,包括“数据清洗与预处理”、“分词”、“特征提取”、“特征表示”、“算法应用”等。并分别就短文本分
类和短文本数据流分类介绍相关的技术。

3. 第三章为本文核心,介绍了基于概念漂移检测的短文本数据流分类算法。该算法大致思想是:通过文本扩展,缓解语义信
息缺失,借助时间序列对数据流进行块分割,对每个块训练基分类器构建集成模型,并通过计算块和块语义距离判断是
否发生了概念漂移。文本扩展使用的Wikipedia作为外部语料,借助LDA主题模型进行主题分析,通过主
题相似性进行文本扩展。

4. 第四章介绍了构建数据挖掘平台的细节,并对前面提到了基于概念漂移检测的短文本数据流分类算
法进行整合。该平台基于Django搭建,采用MVT的开发模式,前后端逻辑分离,使用Scikit-learn、
pandas、numpy、nltk等强大的数据挖掘与机器学习库,完成算法设计。

\section{工作展望}
进入大数据时代的今天,互联网中越来越多的如“Tweets”、“微博”、“新闻标题”等短文本数据,短文本
分类问题将一直将是研究的焦点。短文本数据流带来语义信息不足、特征高维稀疏等问题,本文提出借
助外部语料库扩展的方式,缓解稀疏性,实验表明具有良好的效果。同时,本文提出的数据挖掘平台当
前较为简单,经过进一步研究认为,未来还有一下几个方面的工作值得进行:

\begin{itemize}
\item 本文通过外部语料库扩展的方式缓解文本特征高维稀疏的问题,虽然方法可行,但是扩展过程较
  为耗时,考虑是否可以使用更好的方式对文本空间进行扩展。
\item 对于短文本数据流的分类问题,本实验采用的数据集较小,取得的效果无法证明实际应用大数据的
普适性,接下来需要继续研究如何将算法应用到真实的大数据集当中。
\item 数据挖掘平台的优化,文本仅应用了一种基于集成学习的分类方法到数据挖掘平台当中,在后续研究中,考虑添加更多的数据挖掘算法到平台中。同时,平台的数据需使用手动上传的方式进行模拟,未来考虑整合API或者爬虫的方式,对数据进行实时分析和预测。
\end{itemize}

总之,短文本数据流分类作为数据挖掘研究的重要方向,可以研究的内容还很多,在接下来的学习中,如何将算法真正应用到实际中,提供优雅的解决方案,始终是研究者的重要任务。